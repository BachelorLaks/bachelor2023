\section{Results and discussion}\label{sec:results}
The following chapter will present the results of the models presented in Chapter~\ref{sec:methodology} before comparing the different models and discussing the results. We will start by examining the different ARIMA models, from the simplest ARIMA to the more complex SARIMA. We will then compare this to the LSTM model. 

\subsection{ARIMA}\label{sec:arima}
As explained in section \ref{ARIMA and SARIMAX Methodology}, determining the order of the ARIMA model can be a difficult and complex task, examining the ACF and PACF plots can be a good starting point, but will also be a bit subjective. The only objective way to optimize the model is to use a loss function. In our case we have chosen root mean squared error as this is a commonly used way to measure the accuracy of a model.

In order to find the model with the least RMSE, we performed a grid search on the different parameters using a nested for-loop and outputting the results in the following table: 
\begin{table}[H]
    \begin{center}
        \import{data/Figures/ARIMA/}{AutoARIMAResults}
        \caption{Results of the grid search for the ARIMA model.}\label{tab:ARIMAResults}
    \end{center}
\end{table}
Examining table~\ref{tab:ARIMAResults} we can see that the ARIMA(0,1,1) has the lowest RMSE and is therefore expected to be the best model, but there is no significant difference between the models, this may indicate that the model is not able to capture the trends. An ARIMA model with $p$ of 0, $d$ of 1 and $q$ of 1 is, according to \textcite{nau_2019}, a simple exponential smoothing model which indicates that it may capture the moving average trend, but no other trends. Fitting the model to the data and plotting the results against the actual values we get the following plot:
\begin{figure}[H]
    \begin{center}
        \includegraphics[width=0.9\textwidth]{data/Figures/ARIMA/ARIMA_0_1_1.png}
        \caption{ARIMA(0,1,1) model fitted to the data.}\label{fig:ARIMA_011}
    \end{center}
\end{figure}
As we can see, the model was not able to capture the trend, it seems that it was only able to take the average and draw it further out. In addition, the 95\% confidence interval is very wide, indicating that the model is less than accurate.

Since this was the ARIMA model with the lowest RMSE, there is no reason to believe that the other ARIMA models will be any more accurate. We will therefore not examine the other ARIMA models, but instead move on to the SARIMA models.

\subsection{SARIMA}\label{sec:sarima}
One of the more prominent features of the Salmon data is the clear seasonality that is exhibited on a yearly basis. We should therefore expect a seasonal model to better be able to capture this trend. As we did with the ARIMA models, we will perform a grid search on the different parameters using a nested for-loop, the problem with this is that the SARIMA model has 6 parameters instead of 3. The search will therefore grow exponentially. Consequently, we decided to solely use a $P$ of 0, $D$ of 1 and $Q$ of 0 as a larger $P$ and $Q$ seemed to have a negative effect on the RMSE. As the data has a seasonality of 52 weeks, this is the seasonal parameter we will use. The ten best results of the grid search are presented in the following table:
\begin{table}[H]
    \begin{center}
        \import{data/Figures/ARIMA/}{AutoSARIMAResults}
        \caption{Results of the grid search for the SARIMA model.}\label{tab:SARIMAResults}
    \end{center}
\end{table}
Similarly to the ARIMA model, there is not a significant difference between the models, but the RMSE is clearly lower in the SARIMA than the ARIMA, this indicates the importance of capturing the seasonal trend in the dataset. The SARIMA(2,1,0)(0,1,0)[52] model has the lowest RMSE and should therefore be the most accurate SARIMA model. After fitting the model on the train data and comparing the predictions against the actual data we get the following plot:
\begin{figure}[H]
    \begin{center}
        \includegraphics[width=0.9\textwidth]{data/Figures/ARIMA/SARIMA_2_1_0_0_1_0_52.png}
        \caption{SARIMA(2,1,0)(0,1,0,52) model fitted to the data.}\label{fig:SARIMA_21001052}
    \end{center}
\end{figure}
As we can see in figure~\ref{fig:SARIMA_21001052}, and as predicted from the RMSE, the SARIMA model is able predict the data much closer to the actual data than the ARIMA model. The 95\% confidence interval is still quite large, especially when we reach the end of 2022, but as we can see from the large spike in the spring of 2022, this interval is necessary to be accurate. We can also take a closer look at the twenty first predictions for the year 2022:
\begin{table}[H]
    \begin{center}
        \import{data/Figures/ARIMA/}{SARIMAForecastTable2}
        \caption{SARIMA(2,1,0)(0,1,0,52) model predictions for the year 2022.}\label{tab:SARIMA_forecast}
    \end{center}
\end{table}
Sorting the difference between actual and predicted values in ascending order we can see that the model does have some large errors, especially in the spring of 2022 with the largest difference being -33.5. We can also take a look at the actual model with the different parameters:

\begin{table}
    \begin{center}
        \begin{tabular}{lclc}
        \toprule
        \textbf{Dep. Variable:}          &          SalmonPrice           & \textbf{  No. Observations:  } &    457      \\
        \textbf{Model:}                  & SARIMA(2, 1, 0)x(0, 1, 0, 52) & \textbf{  Log Likelihood     } & -1190.4   \\
        \textbf{Date:}                   &        Fri, 21 Apr 2023        & \textbf{  AIC                } &  2386.72   \\
        \textbf{Time:}                   &            14:19:24            & \textbf{  BIC                } &  2398.73   \\
        \textbf{Sample:}                 &           04-07-2013           & \textbf{  HQIC               } &  2391.48   \\
        \textbf{}                        &          - 01-02-2022          & \textbf{                     } &             \\
        \textbf{Covariance Type:}        &              opg               & \textbf{                     } &             \\
        \bottomrule
        \end{tabular}
        \begin{tabular}{lcccccc}
                        & \textbf{coef} & \textbf{std err} & \textbf{z} & \textbf{P$> |$z$|$} & \textbf{[0.025} & \textbf{0.975]}  \\
        \midrule
        \textbf{ar.L1}  &       0.1953  &        0.044     &     4.486  &         0.000        &        0.110    &        0.281     \\
        \textbf{ar.L2}  &      -0.2932  &        0.043     &    -6.895  &         0.000        &       -0.376    &       -0.210     \\
        \textbf{sigma2} &      21.2106  &        1.328     &    15.966  &         0.000        &       18.607    &       23.814     \\
        \bottomrule
        \end{tabular}
        \begin{tabular}{lclc}
        \textbf{Ljung-Box (L1) (Q):}     & 0.14 & \textbf{  Jarque-Bera (JB):  } &  6.15  \\
        \textbf{Prob(Q):}                & 0.71 & \textbf{  Prob(JB):          } &  0.05  \\
        \textbf{Heteroskedasticity (H):} & 2.14 & \textbf{  Skew:              } & -0.14  \\
        \textbf{Prob(H) (two-sided):}    & 0.00 & \textbf{  Kurtosis:          } &  3.54  \\
        \bottomrule
        \end{tabular}
        \caption{SARIMA Results}\label{tab:SARIMA_results}
    \end{center}
\end{table}
We can in table~\ref{tab:SARIMA_results} see our two autoregressive terms, ar.L1 and ar.L2, and the error term sigma2. The important thing to note from the table is that both the autoregressive terms has a p-value of 0.000, which means that they are both statistically significant. Another important conclusion to draw from the SARIMA results is whether or not the residuals are independent, or white noise. Examining the Ljung-Box test, we see that it produces a result with a p-value of 0.71, this is far greater than the critical value of 0.005. This means that we cannot reject the null hypothesis that the residuals are independent, and there could therefore be more information in the residuals that the model was not able to capture. 

While the AIC and BIC values both can be important to when comparing models, the change in differencing and seasonality makes it difficult to compare the models purely based on this criterion. This is part of the reason why we chose to use the RMSE as our main criterion for comparing the models.
\subsection{SARIMAX}\label{sec:sarimax}

\subsection{LSTM}\label{sec:lstm}

\subsection{Comparison}\label{sec:comparison}

\subsection{Discussion}\label{sec:discussion}