\section*{Sammendrag}
\addcontentsline{toc}{section}{Sammendrag}
Formålet med denne oppgaven er å bruke analytiske modeller til å predikere lakesprisen ved hjelp av lignende råvarer og økonomiske faktorer. For å produsere disse prediksjonene har vi brukt torskeprisen, kveiteprisen, KPI og TWI. Vi har utviklet fire modeller; ARIMA, SARIMA, SARIMAX og LSTM. Disse er alle trent på et datasett samlet inn fra Norsk Råfisklag, Norges Bank, Fishpool og SSB. Modellene er trent på data fra tidsperioden vår 2013 til vinter 2021, og testet på data fra 2022.

Vår forskning og testing har vist at prediksjonene forbedres når sesongvariasjon tas i betraktning. Forbedringen fra ARIMA- til SARIMA-modellen er vesentlig, mens forbedringen fra SARIMA til SARIMAX med én eksogen variabel er veldig liten. Dette vises også tydelig i prediksjonene fra de tre modellene, der SARIMA og SARIMAX omtrent følger de samme trendene, mens ARIMA predikerer den samme gjennomsnittsprisen for alle uker. Den beste prediksjonen fra LSTM-modellen er også nærme RMSE'en til SARIMAX-modellen, men resultatene fra LSTM-modellen varierer stort avhengig av "look back"-perioden. "Look back" på 52 og 104 uker er best på å fange trendene, men "look back"-periodene for LSTM med lavest RMSE er tilfeldig. Med kort "lookback"-periode predikeres gjennomsnittet relativt bra. Dette resulterer i lav RMSE, men ved å se på grafene er det tydelig at dette ikke representeterer en god prediksjon. 

Selv om SARIMAX-modellen presterer best, er det fortsatt områder med usikkerhet. Tester på SARIMAX-modellen viser at vi ikke kan konkludere med om residualene er støy. Den eneste konklusjonen vi kan trekke er at vi ikke kan forkaste nullhypotesen om at residualene er støy.

Konklusjonen fra denne oppgaven er at modellene som tar høyde for sesongvariasjon til en viss grad klarer å følge trendene, men de klarer ikke å predikere lakseprisen med høy nøyaktighet. 

