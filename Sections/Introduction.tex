\section{Introduction}

In this thesis we would like to answer the question. 
Can we predict the pricing of salmon using similar commodities and macroeconomic factors? To answer this we first need to look at fish as a commodity. "As the largest traded food commodity in the world, seafood provides sustenance to billions of people world wide. More then 3 billion people in the world rely on wild-caught and farmed seafood." (\cite{wwf_2019}) Seafood has been a traded commodity for hundreds of years, and for most of this time has been one of Norway's biggest exports. "Norwegian clipfish has been exported to Southern Europe and beyond since the early 1700s." (\cite{seafood_from_norway_clipfish}) Fish world wide has been an important source of protein. Faceing an ever larger growing population and the sustainability issues that comes with it, the seafood industry plays a major role in solving these issues.

For Norway the biggest contribution to helping with this issue is the salmon farming industry. "The salmon industry is one of the biggest industries in Norway." (\cite{Johansen_et_al_2019}) The companies in the industry impact the rest of the Norwegian economy and society as a whole through labor and culture. This in turn means that the Norwegian society is indirectly effected by the the price of salmon on the open market. Salmon export makes up about 2/3 of the Norwegian seafood export by value, amounting to a total of more than 105 billion NOK in 2022. This made salmon export the third largest exported commodity by value in Norway, below oil and gas. (\cite{e24_gasprice_2023}) (\cite{seafood_nokkeltall})

Most of the salmon that is exported comes from fish farms. Salmon farming is a Norwegian lead global industry with four of the 5 biggest companies being Norwegian. (\cite{ilaks_2020}) These companies all rely heavily on the price of salmon. We hope that by creating predictive models on the price of salmon short term we will gain insight into the near future of these major companies. This in turn could give an idea of the impact they might have, and the impact a changing salmon price have on the Norwegian economy and society as a whole.

Fishing as a trait is seasonal given that different fish wanders and breads at different times throughout the year. Although farmed fish are not subjected to this seasonality society has adjusted their consumer habits to this seasonality of commodities long before fish farms and we suspected that this is the case for the salmon to. The quantitative data for sales also support this notion.

As bussiness analysts we found this reaserch question interesting because it enables us to utilize  methods and logic form courses we have preaviously compleated.
The theme of the reaserch question is also highly relevant in the economy today for multiple reasons. Large actors in the economy is just now starting to implement the major advances in computation such as AI and other Artificial neural networks. This is something we will be using to see if it can help our prediction. The fish farming industry is a cornerstone in the Norwegian economy and has been touted as one of the ways for Norway to prosper after the oil is gone. The market and industry surrounding salmon has also been a subject of debates as of recently. This debate comes as a result of massive growth, profits and old tax-related incentives for fish farming. This has caused a lot of turbulence in the price of the salmon farming companies. In this thesis we will not be tackling the issues of the salmon tax debates and its impact on the industry. But we will be examining the changes in price that happened during the hight of this debate. We hope that by the end of this thesis we will have improved our skills in the field of bussiness analytics and have gained a better understanding of the fish farming industry.
