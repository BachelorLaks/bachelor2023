\section{Introduction}
%Ikke tenk for mye på subsection og utforming, i hovedsak bare sånn for at det skal være overskitlig å skirve på foreløpig
Start of Introduction

1. Our problem: “Predicting the pricing of salmon using similar commodities and macroeconomic factors” 
 
Why is our problem important and what is its relevance? 

\subsection{Fish as a commodity}
"As the largest traded food commodity in the world, seafood provides sustenance to billions of people world wide. More then 3 billion people in the world rely on wild-caught and farmed seafood." (\cite{wwf_2019}) Seafood has been a traded commodity for hundreds of years, and for most of this time has been one of Norway's biggest exports. "Norwegian clipfish has been exported to Southern Europe and beyond since the early 1700s." (\cite{seafood_from_norway_clipfish}) Fish world wide has been an important source of protein. Facing an ever larger growing population and the sustainability issues that comes with it, the seafood industry plays a major role in solving these issues.

\subsection{Why salmon?}
"The salmon industry is one of the biggest industries in Norway." (\cite{Johansen_et_al_2019}) The companies in the industry impact the rest of the Norwegian economy and society as a whole through labor and culture. This in turn means that the Norwegian society is indirectly effected by the the price of salmon on the open market. Salmon export makes up about 2/3 of the Norwegian seafood export by value, amounting to a total of more than 105 billion NOK in 2022. This made salmon export the third largest exported commodity by value in Norway, below oil and gas. (\cite{e24_gasprice_2023}) (\cite{seafood_nokkeltall})

\subsection{Fish farming}
Most of the salmon that is exported comes from fish farms. Salmon farming is a Norwegian lead global industry with four of the 5 biggest companies being Norwegian. (\cite{ilaks_2020}) These companies all rely heavily on the price of salmon. We hope that by creating predictive models on the price of salmon short term we will gain insight into the short term future of these major companies. This in turn could give us an idea of the impact they might have on the Norwegian economy and society as a whole in the short term.

\subsection{Is this relevant from a business analytics standpoint?}
This thesis is relevant in Business analytics as we will be creating a model using methods and logic form previous courses we have had. Its highly relevant in the economy today for multiple reasons. For one the Industry is a cornerstone in the Norwegian economy and has been touted as one of the ways for Norway to prosper after the oil is gone. The market and industry surrounding salmon has also been a subject of debates as of recently. This debate comes as a result of massive growth, profits and old tax-related incentives in the industry. This has caused a lot of turbulence in the price of the salmon farming companies. In this thesis we will not be tackling the issues of the salmon tax debates and its impact on the industry. But we might have to make adjustments in our data set because of it.

\subsection{Hva ønsker vi å oppnå med thesisen?}
Skrive litt om hva vi ønsker å finne ut av eller å finne ut?
Hva som gjør det vanskelig å finne ut av?
Noen statistiske faktorer med datasettet vårt som random sampling osv.

Kilde 2: https://www.worldwildlife.org/industries/sustainable-seafood
Kilde 3: https://fromnorway.com/seafood-from-norway/clipfish/
Kilde 4: https://e24.no/norsk-oekonomi/i/xgOd7X/skyhoeye-gasspriser-ga-historisk-hoey-eksport-i-2022 (ikke direkte sitert i avsnittet, men)
Kilde 5: https://nokkeltall.seafood.no (ikke direkte sitert i avsnittet, men)
Kilde 6: https://ilaks.no/dette-er-verdens-20-storste-lakseoppdrettere-2/

Alle disse lagt inn under references