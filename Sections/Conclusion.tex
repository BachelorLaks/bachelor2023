\section{Conclusion}
\begin{itemize}
    \item Start with the research question
    \item Continue with repeating the path of the research
    \item Short summary of the results and which model performed the best
    \item End with why we may not get the results we sought
    \item "It is important to further improve the model to get better results"
\end{itemize}

In this thesis we have predicted the salmon price using different analytical models. In doing so we have tested how well these different models perform, when trying to predict the price of salmon using similar commodities and macroeconomic factors. The approach we had to solving this task was to first perform a simple exploratory analysis to see if there was any changes needed to be done in the preprocessing stage. This was not the case. We then built four different models; ARIMA, SARIMA, SARIMAX and LSTM.

The general findings is that the SARIMA and SARIMAX models produces quite similar predictions, and both outperform the ARIMA model. This clearly indicates that seasonality must be taken into account. When examining the SARIMAX predictions we see that the confidence interval is rather large. However, the salmon price spiked massively in the spring of 2022, so this is necessary. Therefore, simply examining whether or not the model predicts the correct movement of the price is of relevance. Although the model is not great, it is somewhat capable of catching the trends in the short term. The LSTM model seems not to be able to catch the seasonality before it reaches 52 Time steps. This makes sense as we saw from the exploratory analysis that the seasonality is a yearly pattern. But when looking back at 52 weeks or more then models gets better at catching the seasonality as well as other trends.

Overall the SARIMAX model seems to be the best model for predicting the salmon price. However, the LSTM model has the potential to be the best model, but it needs more work. The struggles of the best models were mainly predicting the big magnitude in price change as it is following the price change in direction in many cases, but not as steep as the actual change.

Examining the Ljung-Box test on the SARIMAX model shows that the residuals are not white noise, and that there might be some trends left to capture. We therefore suggest that future research focuses on improving the models and perhaps lenghten the dataset in order to get better odds of picking up these more illusive trends. One posibility for imporving the models is simply to use more computing power as this would allow for more complex models. Another possibility is to use different variables in the multivariate analysis in LSTM as we saw the multivariate was by far the best version of LSTM. This might also apply to the SARIMAX model. It is possible to try different models entierly, but we suggest focusing on multivariate.
